\documentclass[11pt]{amsart}

\usepackage{macros, setspace}

\author{Owen Gwilliam and Brian R. Williams}
\date{\today}
\title{Part I}

\def\g{{\mathfrak g}}

\def\brian#1{{\textcolor{blue!65!red}{BRW: {#1}}}}
\def\owen#1{{\textcolor{green!65!black}{BRW: {#1}}}}

%\usepackage[upint]{stix}

\begin{document}
\maketitle

\spacing{1.25}


%"classical field theory", don't mention BV.

%the way we discussed previously, this section seemed to be mostly geometric.

%Motivational overview

\section{What is a holomorphic field theory}

The most important field theories in physics involve a metric on a manifold (whether Riemannian or Lorentzian), and hence geometry has had a clear role in physics.
In recent decades attention has expanded to include {\em topological} field theories (or TFTs), 
in which --- to simplify --- only the underlying smooth topology of the manifold matters.
There are, however, theories that depend on a complex structure on the manifold,
and as one might hope, these are particularly beautiful, 
just as complex variables is a particularly beautiful wing of analysis.
In essence, such {\em holomorphic} field theories have variational PDE (or ``equations of motion'') that are holomorphic:
the PDE only involves derivatives in the holomorphic coordinates (i.e., the $\partial/\partial z_j$) and has holomorphic coefficients.
These theories admit independent motivations from mathematics and physics,
which we sketch before giving more careful definitions.

The mathematical motivation is simple:
for most classical holomorphic field theories, the solutions to the equations of motion form a moduli space of natural interest to complex geometers,
and it is natural to hope that its quantization is equally interesting.
A well-known example is holomorphic Chern-Simons theory,
which lives on a Calabi-Yau 3-fold $X$ and involves the {\em derived} moduli space $\RR\Bun_G(X)$ of holomorphic principal $G$-bundles on $X$.
Other classes of examples are afforded by holomorphic variants of familiar AKSZ $\sigma$-models; the relevant moduli space being the derived mapping space between complex manifolds. 

A physical motivation comes from supersymmetric field theories, 
which (loosely speaking) are theories on $\RR^n$ whose symmetries include not only the isometries of $\RR^n$, but an extension to a Lie supergroup known as a super Poincar\'e group.
%The relevant extension is often labeled by $\cN=1, 2,$ and so on, 
%with larger $\cN$ denoting a bigger supergroup. 
Thanks to the enhanced symmetry, such theories can often be understood in greater detail than non-supersymmetric theories.
The relevance to the models we study is that supersymmetric theories admit ``twists'' (in essence, deformations) that are holomorphic field theories, 
even when they do not admit twists to TFTs (such as the famed A- and B-models of mirror symmetry) \cite{CosHol}.
As an example, four-dimensional minimally supersymmetric Yang-Mills theory admits no topological twist, but does admit a twist to holomorphic field theory whose moduli space is described, in part, by $\RR \Bun_G(X)$. 
It is natural to wonder how the myriad tools and computations deployed on the supersymmetric theory carry over to this twist and conversely whether information can flow back from a holomorphically twisted theory to the untwisted one, 
much as TFTs have enriched the understanding of supersymmetric theories.
In short, a holomorphic twist contains the ``holomorphic sector'' of the untwisted theory just as a topological twist contains its ``topological sector.''

As a final motivational remark, we note in one complex dimension (i.e., working on Riemann surfaces), holomorphic field theories appear as chiral conformal field theories (CFTs) on 2-dimensional manifolds --- whose manifestations in mathematics include vertex algebras, affine Lie algebras, and loop groups --- and have had a strong impact in mathematics.
It is natural to explore higher-dimensional analogs.

The overarching goal of this paper is \owen{continue}


cluster of projects is to bring into sharp mathematical form some of the deep insights into gauge theory offered by physics.
So far, most mathematical effort has focused on classical gauge theory,
but the quantum aspects are profoundly interesting.
We expect that these holomorphic examples will be particularly tractable and beautiful but also useful in offering insights even for non-holomorphic theories.

\subsection{Preliminary notions}

%complex geometry, dolbeault, dg stuff.

\subsection{Precise definitions}

Let us begin with the example of holomorphic Chern--Simons theory, which we have already said describes the moduli space of 

Maurer--Cartan elements describe deformations of the trivial holomorphic $G$-bundle. 
When $X$ is a Calabi--Yai three-fold, and $\fg$ is equipped with an invariant pairing, this formal moduli space has a very special property: it arises from the variational problem of an action functional, namely
\[
S = \int_X \Omega \wedge {\rm CS}(A) 
\]
where $\Omega$ is the holomorphic volume form and ${\rm CS}(A)$ is the Chern--Simons Lagrangian for the connection $(0,1)$-form $A$.

We turn to a formal definition underpinning the formal moduli spaces of the holomorphic field theories that we will be considering. 

\begin{dfn}
A {\em holomorphic dg Lie algebra} on a complex manifold $X$ is the data:
\begin{itemize}
\item A graded holomorphic vector bundle $V$ on $X$ whose sheaf of holomorphic functions is $\cV^{hol}$. 
\item A holomorphic differential operator $Q^{hol} \colon \cV^{hol} \to \cV^{hol}$ of cohomological degree one and a holomorphic bidifferential operator $[\cdot,\cdot] \colon \cV^{hol} \times \cV^{hol} \to \cV^{hol}$. 
\end{itemize}
This data is required to endow $(\cV^{hol}, Q^{hol}, [\cdot,\cdot])$ with the structure of a dg Lie algebra.
\end{dfn}

Any holomorphic vector bundle has an associated Dolbeault complex $\Omega^{0,\bu}(X,V)$. 
If $V$ is a holomorphic dg Lie algebra then $\Omega^{0,\bu}(X, V)$ has the induced structure of a dg Lie algebra. 
The differential is of the form $\dbar + Q^{hol}$ and the bracket is given by extending the bracket on $\cV^{hol}$ with the wedge product of differential forms. 

Formal moduli spaces arise from the space of Maurer--Cartan elements of a dg Lie algebra.
In our case, this dg Lie algebra will always be of the form $\Omega^{0,\bu}(X, V)$.
For example, the Maurer--Cartan elements of the dg Lie algebra $\Omega^{0,\bu}(X, \fg)$ recovers formal deformations of the trivial holomorphic $G$-bundle.  
This arises from the holomorphic Lie algebra on $X$ whose underlying vector bundle is trivial with fiber $\fg$. 

As another example, consider the holomorphic tangent bundle $V = \T_X$. 
The Lie bracket of vector fields endows this with the structure of a holomorphic Lie algebra (with zero differential). 
A Maurer--Cartan element in $\Omega^{0,\bu}(X, \T_X)$ is an element $\mu \in \Omega^{0,1}(X, \T_X)$ satisfying the equation
\[
\dbar \mu + \frac12 [\mu,\mu] = 0 .
\]
This recovers the classical picture describing formal deformations of complex structures on~$X$.



%this section aimed at mathematicians
%emphasize connection with (derived) complex geometry

\subsection{Examples from physics}
%this section aimed at physicists
%twists of SUSY theories
%Owen has some nice thoughts about globalization as it related to SUSY/THF.


Any even dimensional supersymmetric field theory produces a holomorphic field theory through a process called twisting,
so we have a wealth of examples to study that are intimately connected to theories of genuine interest in physics.\footnote{Strictly speaking, this requires the existence of of more than two chiral supercharges. So, theories with 2d $\cN = (n,m)$ supersymmetry, $n,m \leq 1$, do not admit holomorphic twists.}
As we will explain later, many interesting phenomena for supersymmetric theories have analogs in holomorphic field theory.
 
By definition, a supersymmetric field theory on $\RR^d$ is a theory that is acted upon by a super Poincar\'{e} algebra. 
We focus on Euclidean field theories and work in Riemannian signature,
hence for us the ``super Poincar\'{e} algebra'' is a super Lie algebra of the form
$\mathfrak{so}(d) \ltimes \ft$
where $\ft$ is the super Lie algebra of {\em supertranslations} whose even part $\ft^0 = \RR^d$ is the Lie algebra of ordinary translations and whose odd part $\ft^1$ is a sum of spin representations. 
While the Lie algebra $\RR^d$ of ordinary translations is abelian, the Lie algebra $\ft$ carries a nontrivial (super) Lie bracket,
and this Lie bracket is defined in terms of a $\mathfrak{so}(d)$-equivariant non-degenerate symmetric pairing
\[
\Gamma : {\rm Sym}^2(\ft^{1}) \to {\rm Sym}(\ft^0) \cong \RR^d 
\]
by the formula $[Q, Q'] = \Gamma(Q, Q')$. 

By a {\em supercharge}, one means an odd supertranslation $Q \in \ft$, and
a {\em twist} is a (nonzero) supercharge $Q$ such that $[Q,Q] = Q^2 = 0$. 
The classification of all twists is a completely algebraic question,
and we refer to~\autocite{ESsusy} for a complete classification of all twists in dimensions from $1$ to~$10$. 

If a Lagrangian field theory $\cT$ has $\ft$ as a symmetry, 
then a choice of twist $Q$ determines a deformation of the theory that we call a {\em twisted supersymmetric theory} $\cT^Q$.
If $S$ denotes the action functional of $\cT$, then the action functional $S^Q$ of $\cT^Q$ has the form
\[
S^Q(\varphi) = S(\varphi) + \int_{\RR^d} \varphi  \left(Q \cdot \varphi \right) + \cdots 
\]
where the deformation arises from how $Q$ acts on the theory 
(the $\cdots$ leaves room for terms non-linear in $Q$).
For extensive details on how the twisted theory is defined we refer to~\autocite{CostelloHolomorphic, ESW}. 

Building upon the physical literature, Elliott, Safronov, and the second author have given a complete characterization of the all twisted supersymmetric Yang--Mills theories in dimensions $2 \leq d \leq 10$ \autocite{ESW}. 
These twisted theories are analogs of BF theory and Chern--Simons theory, so long as one includes purely topological and purely holomorphic theories. 
We will describe a few examples in detail in the BV formalism in Section~\ref{sec:twistedsusybv},
but we want to indicate now what our main theorem implies about some interesting examples.

For our purposes here, a crucial property of a twist $Q$ is the dimension $k_Q$ of its image ${\rm Im} \; \Gamma(Q, \cdot) \subset \RR^d$,
which we call the number of {\em invariant} directions of~$Q$.
By the non-degeneracy of the pairing $\Gamma$, it follows that $k_Q \geq \frac{d}{2}$.
If the number of invariant directions is maximal with $k_Q = d$, then the twisted theory $\cT^Q$ is purely topological. 
If the number of invariant directions is minimal with $k_Q = \frac{d}{2}$, then the twisted theory $\cT^Q$ is purely holomorphic. 

As a representative example of how twisting behaves, 
consider four-dimensional supersymmetric Yang--Mills theory with $\cN=2$ supersymmetry,
which means the odd part $\ft^1$ is eight-dimensional. 
%The sum of two copies of the spin representations.
%\owen{Did I get that right?}
We only consider here the pure gauge theory with Lie algebra~$\fg$. 
Then we have the bounds $2 \leq k_Q \leq 4$, and each value gives a different, well-known field theory:
\begin{itemize}
\item When $k_Q = 4$, the twisted theory is the topological field theory on $\RR^4$ known as {\it Donaldson--Witten theory} \autocite{WittenTQFT};
\item When $k_Q = 3$, the twisted theory is the THFT studied by Kapustin in \autocite{Kapustin};
\item When $k_Q = 2$, the twisted theory is holomorphic BF theory on $\CC^2$ with values in a graded Lie algebra $\fg[\ep]$, where $\ep$ is a parameter of cohomological degree~$1$.
\end{itemize}
We will describe these theories completely later in Section~\ref{sec:twistedsusybv},
but our main theorem implies the following.

\begin{thm}[\owen{cite GRW, maybe also mention our work with Chris?}]
Donaldson--Witten theory, the Kapustin twist, and holomorphic BF theory for $\fg[\epsilon]$ each admit an exact and finite quantization at one loop.
\end{thm}

In other words, by a judicious choice of gauge-fixing, 
we find that the one-loop Feynman diagrams have no UV divergences.
Moreover, no higher loop diagrams appear (for combinatorial reasons),
and the action functional satisfies the quantum master equation.
See Section~\ref{sec: susytwist revisited} for some further discussion.



\subsection{Background}

Although holomorphic field theories had already appeared in practice,
my collaborator Brian Williams was the first to lay a systematic, mathematical foundations (cf. his thesis \cite{BWthesis} and its offshoot \cite{BWhol}).
He did the following:
\begin{itemize}
\item gave a precise definition of a holomorphic field theory in the BV formalism,
modeled on and compatible with Costello's approach;
\item established key analytic results about holomorphic renormalization, 
showing that it is highly manageable, with no counterterms to 1-loop; and
\item characterized the one-loop anomalies  (i.e., obstructions to BV quantization).
\end{itemize}
These apply to all holomorphic theories on $\CC^n$,
but the techniques typically extend to more general complex manifolds.

This foundational work was motivated in large part by our joint effort to generalize to complex $n$-folds results about chiral conformal theories on Riemann surfaces.
A first fruit of our work \cite{GWcurr} is an examination of higher-dimensional {\em current algebras},
which are factorization algebras that generalize the Kac-Moody vertex algebras that play a crucial role in contemporary representation theory.
We show, for instance, that these current algebras arise as the quantization of symmetries of holomorphic field theories, 
much as the affine Lie algebras are symmetries of CFTs on Riemann surfaces.
In addition, we show that encoded in our factorization algebras are the higher Kac-Moody Lie algebras $\widehat{\g}_{d,\theta}$ of Faonte-Hennion-Kapranov~\cite{FHK}. 
It is striking to find this rigid skeleton of algebra under the meat of our analysis,
as our work is strongly differential-geometric in flavor.
Moreover, as their work provides a direct connection with the derived algebraic geometry of $\RR Bun_G(X)$,
there emerges a beautiful generalization of the fertile interactions between algebraic geometry, representation theory, and physics that have grown around bundles on Riemann surfaces.
(The prospect looms that in higher dimensions,
one might find a correspondence between conformal blocks and nonabelian theta functions.)

A second fruit is a collaboration of Williams with the physicist Ingmar Saberi \cite{SabWil1,SabWil2}.
They have carefully worked out the holomorphic twists of several supersymmetric theories
and given a novel approach, via factorization algebras, to the results of Beem et al \cite{Beem}.
Particularly important for this proposal,
they have shown how important invariants (notably the ``supersymmetric index") of the untwisted, supersymmetric theory can be computed simply in the holomorphic twist.

\subsection{More examples of holomorphic field theories}

\begin{itemize}
\item Target space field theories of topological string theories give rich classes of holomorphic field theories. 
For instance, the closed string field theory of the topological B-model is Kodaira--Spencer theory, first proposed in \cite{BCOV}. 
This theory is defined on any Calabi--Yau manifold. 
The perspective of Kodaira--Spencer theory as a holomorphic field theory has been explored with great success by Kevin Costello and Si Li \cite{CL1, CL2, CL3}.  
\item The `pure spinor' method of Cederwall and Berkovits \cite{Cederwall, Berkovits} starts with a field or string theory defined on the (sometimes singular) nilpotence variety of the super Poincar\'e algebra and uses the action of supersymmetry to produce ordinary (non-twisted) supersymmetric field theories. 
Often, the theory you start with on the nilpotence variety is holomorphic \cite{SWpure}. 
\item Similar in spirit to the pure spinor formalism is a systematic relationship between holomorphic field theories on (super) twistor space and ordinary (Riemannian) field theories.
For instance, holomorphic BF theory on twistor space associated to $\RR^4$ is equivalent to a self-dual limit of Yang--Mills theory. 
\end{itemize}

\end{document}