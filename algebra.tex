\documentclass[11pt]{amsart}

\usepackage{macros, setspace}

\def\U{{\rm U}}
\def\mc{\mathcal}
\def\mcol{\, | \,}
\def\ot{\otimes}
\def\Disj{\operatorname{Disj}}
\def\Open{\operatorname{Open}}
\def\Vect{\operatorname{Vect}}
\def\pAA{\mathring{\AA}}
\def\jou{{\mathtt{j}}}

%trying this out
%\usepackage[upint]{stix}
%\usepackage{cmupint}

\def\brian#1{{\textcolor{blue!65!red}{BRW: {#1}}}}
\def\owen#1{{\textcolor{green!65!black}{OGG: {#1}}}}


\author{Owen Gwilliam and Brian R. Williams}
\date{\today}

%\title{Part III}

\spacing{1.25}

\begin{document}
%\maketitle

\section{Algebras from holomorphic field theories}

In this section we describe  associative algebras and their ``higher'' generalizations, like $A_\infty$ algebras, related to holomorphic field theories,
following the relationship of associative algebras to quantum mechanics and of vertex algebras to chiral conformal field theory.
A particular focus is upon higher Kac-Moody algebras \cite{FHK},
as these offer a tantalizing direction to explore in search of analogs of the rich connections between representation theory, algebraic geometry, and physics familiar to those who have worked with loop groups.
In a later section --- and it is a central point of this survey --- we will explain how factorization algebras provide a direct conduit from holomorphic field theories to these algebraic constructions.
Throughout this section, we will be a bit cavalier with certain subtleties (e.g., about infinite-dimensional spaces and functions on them), 
emphasizing concepts and motivations over mathematical precision.
We will be a bit long-winded in the first two subsections, to provide motivations,
so that in the final subsection, we can rapidly sketch these higher Kac-Moody and Weyl algebras.

\subsection{Algebras in mechanics}

A key feature of quantum mechanics is that the observables (or operators) live in an associative algebra.
In many cases this associative algebra is a deformation of a commutative algebra, typically arising as functions on a manifold or variety.
The quintessential example is the Weyl algebra
\[
\CC \langle x, p \rangle /(xp - px = i \hbar),
\] 
which is generated by observables $x$ (``position'') and $p$ (``momentum'') for a quantum particle moving along a line.
There is a parameter $\hbar$, which if sent to zero, recovers a commutative algebra $\CC [ x, p ]$ of complex-valued polynomial functions on the cotangent bundle $T^* \RR$ of the real line~$\RR$.
This example will be a model for much of what we discuss in this paper.
So far we have ignored a lot of features of quantum mechanics (e.g., $\ast$-structures, Hilbert spaces, unitarity) that are important in physics,
and we will continue to do so.
Note as well that there are many more observables and we have only discussed subalgebras of all observables (at the classical level, just polynomials in $p$, $q$);
we will often focus on such tractable subalgebras. 

There are some features of this example that we would like to foreground. 
First, the commutative algebra arises as functions on a {\em symplectic} space,
which is the usual mathematical setting for classical mechanics (aside from more subtle situations that require Poisson geometry).
Second, the symplectic form $\omega = \d x \wedge \d p$ equips this algebra 
with a Poisson bracket where $\{ x, p \} = 1$,
which controls the deformation to the Weyl algebra: following Dirac, we promote the Poisson relation to a commutator relation.
These two features motivate the deformation quantization problem: 
given a symplectic (or Poisson) manifold,
describe deformations of its commutative algebra of functions to an associative algebra with the requirement that, to first order,  the commutator recovers the Poisson bracket.
Thanks to Kontsevich \cite{KonDQ}, there is a beautiful answer to this question, which has spawned a mountain of fascinating mathematics (see, e.g., \owen{what?} as a starting place).

Our view on field theory is motivated by this perspective on quantum theory (we discuss it further in Section~\owen{ref the general overview}), 
and it might help the reader to bear in mind a variant of this question:
given a classical holomorphic field theory, what are the natural deformations of its algebra of functions?
Many of the results we discuss later can be seen as generalizations of phenomena associated with deformation quantization.

In particular, consider how symmetries appear in mechanics.
Let $X$ be a symplectic manifold encoding the ``phase space'' of a classical mechanical system,
and let $\{-,-\}$ denote the Poisson bracket on $C^\infty(X)$.
If a Lie group $G$ acts on $X$ by symplectomorphisms (i.e., respects the symplectic structure),
there is a map of Lie algebras $\rho: \fg \to {\rm SympVect}(X)$ so that an ``infinitesimal symmetry'' $x \in \fg$ acts by a symplectic vector field.
In many cases, such infinitesimal symmetries are realized as observables: 
there is a map of Lie algebras $H_\rho: \fg \to C^\infty(X)$ such that $\{ H_\rho(x), - \} = \rho(x)$.
One says that each element $x$ has a Hamiltonian function $H_\rho(x)$ whose associated Hamiltonian vector field is $\rho(X)$.
Classic examples include the momenta for $T^* \RR^n$ arising from translation and rotational symmetry.

How does this set-up fit into the deformation quantization view?
First, note that the free commutative algebra $\Sym(\fg)$ has a canonical Poisson bracket by defining $ \{x, y \}   = [x, y]$ for generators $x, y \in \fg$ and extending by the Leibniz rule.
Hence the Hamiltonian map $H_\rho$ extends to a Poisson map $H_\rho: \Sym(\fg) \to C^\infty(X)$.
We can then ask: does this map quantize? That is, can we compatibly deform the domain and range of the maps to interesting associative algebras as well as deforming to a map of associative algebras?
Here it might be useful to recognize that the enveloping algebra $U\fg$ is a natural deformation quantization of $\Sym(\fg)$, thanks to the Poincar\'e-Birkhoff-Witt theorem.
Thus our question might be reformulated to asking whether there is an associative algebra map $H^q_\rho: U\fg \to C^\infty(X)^q$, where $C^\infty(X)^q$ is a deformation quantization, such that $H^q_\rho$ recovers $H_\rho$ in the classical limit.
Such a map is sometimes called a {\em quantum moment map},
as $H_\rho$ is the pullback of (polynomial) functions along a moment map~$\mu_\rho: X \to \fg^*$.

This class of questions appears throughout mathematics,
and it has played a key role in geometric representation theory and the theory of $D$-modules, where $D$ denotes a ring of differential operators.
After all, for any manifold $X$, the algebra of differential operators $D_X$ is a natural deformation quantization of functions on $T^* X$.
Hence, for any $G$-manifold $X$, it is natural to ask whether there is a representation $\fg \to D_X$ deforming the canonical Poisson representation.
In this spirit, we will search for analogues of the kind of mathematics that has grown out of results by Beilinson-Bernstein \owen{cite}, Kashiwara \owen{cite}, and others.

\begin{rmk}
We have only spoken of symplectic manifolds, and emphasized vector spaces,
but fermions play a crucial role in quantum mechanics too.
Here the symmetric algebra on a symplectic vector space (as classical observables) 
is replaced by an exterior algebra on an inner product space,
which can be seen as a symmetric algebra on an {\em odd} vector space
(i.e., a super vector space with purely odd component).
The deformation quantization of the exterior algebra is then a Clifford algebra.
We will set fermions aside for the remainder of this section,
but the discussion below extends easily to them and fermions appear naturally in the study of holomorphic field theories.
\end{rmk}

\subsection{A first step in the holomorphic direction}

What could provide the {\em holomorphic} analog of the story above?
Let us indicate one possible answer before we motivate it.

Let $V$ be a finite-dimensional complex vector space, equipped with a symplectic pairing $\omega$ that is complex-valued.
We can view $V$ as $T^* L$ for some Lagrangian vector subspace $L \subset V$, if we wish.
Then we posit for the relevant phase space, the ``symplectic complex manifold'' $V[z,z^{-1}]$,
by which we mean the $V$-valued Laurent polynomials in a coordinate $z$ on the punctured plane $\CC^\times = \CC -\{0\}$.
The space encodes algebraic maps from the variety $\CC^\times$ to $V$,
so we view it as a kind of algebraic loop space of~$V$.
(There are clearly variations on this idea, such as the holomorphic loop space of all holomorphic maps from $\CC^\times$ to~$V$.)

Observables for this classical theory then consist of algebraic functions on $V[z,z^{-1}]$, 
which happens to be a vector space.
That is, we want to produce a symmetric algebra on the linear dual to this vector space.
Recall that Laurent series $\CC((z))$ provide a model of the linear dual to Laurent polynomials via the residue pairing: if $p(z) = \sum_{n = -i}^j a_n z^n$ is a Laurent polynomial and $f(z) = \sum_{m = -k}^\infty b_m z^m$ is a Laurent series, then the pairing is
\[
(f, p) = \Res_{z=0}(f p) = \sum_{m+n=-1} a_n b_m.
\]
Thus $V^* ((z))$, the $V^*$-valued Laurent series, are a linear dual to $V[z,z^{-1}]$ by combining the residue pairing with the evaluation pairing between $V$ and its linear dual~$V^*$.
The symmetric algebra $\Sym(V^* ((z)))$ thus encodes a class of observables that we will denote $\cO(V[z,z^{-1}])$.
(Again, there are clearly alternative algebras to consider.)
When we quantize, this algebra will be related to a well-known vertex algebra.

There is one last feature we would like to point out, before we turn to explaining our interest in this space and algebra.
The inclusion $\CC^\times \hookrightarrow \CC$ means that any algebraic map  $\phi: \CC \to V$ restricts to an algebraic map $\phi: \CC^\times \to V$.
In our notation, this restriction map is the inclusion $V[z] \hookrightarrow V[z,z^{-1}]$ of $V$-valued polynomials in $V$-valued Laurent polynomials.
In terms of observables, there is a quotient map $\cO(V[z,z^{-1}]) \to \cO(V[z])$, since observables on the loop space restrict to observables on~$V[z]$.
When we quantize, this relationship will produce the underlying vector space (or Fock space, or vacuum module) of the vertex algebra.

Now we will discuss why we might focus on these constructions,
and in what sense they are holomorphic versions of usual mechanics.

Let us start by offering a motivation for the ``answers'' (e.g., $\RR[q,p]$ as classical observables) in the setting of standard classical mechanics.
The model problem in mechanics is to describe a point particle moving in $\RR^n$,
subject to some forces that specify the differential equations governing the particle's motion.
(We call these the ``equations of motion.'')
Newton's law says that this equation is second order, so that a solution (i.e., trajectory) is specified by giving the position and velocity of a particle at one instant in time.

There is another view, which generalizes more naturally to field theories and which is dubbed the Lagrangian formalism.
Here one notes that there is a space of all imaginable trajectories, namely the path space ${\rm Map}(\RR, \RR^n)$, where we view the source $\RR$ as ``time,''
and there is a subspace of trajectories realized by the particle, namely solutions to the equations of motion.
This subspace is, in some sense, the critical set of an action functional; 
the variational calculus provides the relevant mathematical framework.
The space of solutions is naturally isomorphic to $T \RR^n$,
where an isomorphism is given by fixing an instant $t_0$ in time $\RR$ and then sending a solution $\phi$ to the pair $(\phi(t_0), \dot{\phi}(t_0))$, the position and velocity of the solution at~$t_0$.
With a little care, one finds that the variational calculus equips $T \RR^n$ with a natural symplectic form and a symplectomorphism $T\RR^n \cong T^* \RR^n$, with its canonical symplectic form. 
Putting everything together, we have a natural symplectomorphism of $T^* \RR^n$ with the space of solutions.
The observables of the classical system are the algebra of functions on the space of solutions,
and so this isomorphism tells us that functions on $T^* \RR^n$ provide the observables.
The polynomial functions on $T^* \RR^n$ are a subalgebra of all observables, 
and they suffice to distinguish distinct solutions.

This view generalizes nicely to the holomorphic setting,
and it amounts to using the Lagrangian formalism to study the holomorphic field theories we have already introduced.
In this model case, we replace the path space ${\rm Map}(\RR, \RR^n)$ by ${\rm Map}(\CC^\times, \CC^n)$ or, if one wishes, ${\rm Map}(S, \CC^n)$ with $S$ a Riemann surface.
We start with $\CC^\times$ since it has a ``time'' direction given by the radial coordinate ($t$ becomes $r = e^t)$),
letting us view ${\rm Map}(\CC^\times, \CC^n)$ as  ${\rm Map}(\RR, \Map(S^1,\CC^n))$, namely mechanics into the loop space of $\CC^n$.
We replace Newton's equations of motion with a holomorphic version
so that the space of solutions is given by holomorphic maps from $\CC^\times$ into $\CC^n$.
If we wish to focus on a more algebraic version, as we did at the beginning,
we could restrict to the algebraic maps from $\CC^\times$ into~$\CC^n$,
which sit inside the holomorphic maps.

A confession is necessary here, because this replacement is a bit misleading.
The holomorphic version of the model case, known as the free $\beta\gamma$ system,
actually has holomorphic maps into $T^* \CC^n$ as the space of solutions.
The algebraic maps are precisely $T^* \CC^n$-valued Laurent polynomials.

We now turn to quantization, and our phase space will be, for simplicity, the algebraic loop space $V[z,z^{-1}]$ of a symplectic complex vector space $V$.
Following the case of mechanics, we might ask for a deformation of $\cO(V[z,z^{-1}])$ into an associative algebra.
A subtlety here is that $\cO(V[z,z^{-1}])$ does not possess a Poisson bracket, 
due to the  infinite-dimensionality of $V[z,z^{-1}]$.
The naive formulas from functions on a finite-dimensional symplectic vector space only make sense on a subset of observables here.
One workaround is to replace $V[z,z^{-1}]$ with $V((z))$, the $V$-valued Laurent series,
which we view as ``formal loops in $V$.''
There is manifestly an inclusion $V[z,z^{-1}] \hookrightarrow V((z))$ and hence an algebra map $\cO(V((z))) \to \cO(V[z,z^{-1}])$
where 
\[
\cO(V((z))) = \Sym(V^* [z,z^{-1}]).
\]
On this subalgebra, the naive Poisson bracket makes sense: it is given by extending the nondegenerate skew-symmetric pairing $\omega^* \otimes \Res$ on $V^* [z,z^{-1}]$.
Hence for $\cO(V((z)))$, we can ask for a deformation quantization,
and there is a nice answer: 
the Heisenberg commutation relations determine a Weyl algebra $W[V]$ for this formal loop space.
As a vector space $W[V]$ is isomorphic to $\cO(V((z)))$, 
just as the simplest Weyl algebra is isomorphic to $\CC[p,q]$ as a vector space,
but the product structure is modified by quantization.
We expect, by analogy with the simplest case, that any quantization of a bigger algebra (e.g., for observables of the holomorphic loop space) contains this quantization as its algebraic skeleton.

The subspace of ``contractible formal loops" $V[[z]]$ inside $V((z))$ produces a module for $\cO(V((z)))$ given by $\cO(V[[z]]$. 
(This module is in fact a quotient algebra.)
One can quantize this module to a module ${\rm Vac}[V]$ for $W[V]$,
where as a vector space, it is still $\cO(V[[z]])$.
This module structure means there is a map of algebras $W[V] \to \End({\rm Vac}[V])$,
so we have a natural ``Hilbert space'' or Fock module for~$W[V]$.

It is a remarkable fact, suggested by physicists, that there is also a map
\[
Y: {\rm Vac}[V] \to \End({\rm Vac}[V])[[z, z^{-1}]],
\]
known as the {\em vertex operator} or {\em state-field correspondence},
which is closely related to the $W[V]$-action.
The intuition behind $Y$ is that for any map $\phi: \CC^\times \to V$ and for any point $w \in \CC^\times$, 
there is a restriction of $\phi$ to a little disk around $w$.
That is, there is a restriction map $r_w: V[z,z^{-1}] \to V[[t]]$, where $t$ is the local coordinate $t = z-w$.
Hence there is a $w$-dependent map $\cO(V[[t]]) \to \cO(V[z,z^{-1}])$ by pulling back a function along $r_w$;
this map determines a $w$-dependent action of $\cO(V[[t]])$ on $\cO(V[[z]])$.
One can, in essence, quantize this map, by trying to extend the formulas of the $W[V]$-action,
and this leads to the map~$Y$.
If one axiomatizes the behavior of $Y$, one is led to the notion of a vertex algebra.

Let us briefly comment on how symmetries extend to the holomorphic setting.
In other words, we wish to explain how the affine Lie algebras $\widehat{\fg}$ and free field realizations arise in analogy with our discussion of quantum moment maps.

Suppose now that $G$ is a complex Lie group and it acts on the symplectic complex vector space $V$ by holomorphic symplectomorphisms.
Then $G$ also acts on the algebraic loop space $V[z,z^{-1}]$ and also on formal loops $V((z))$ as a ``global'' symmetry:
the action is independent of $z$.
On the other hand, if we consider the infinitesimal action $\rho: \fg \to {\rm SympVect}(V)$,
there is, in fact, a natural extension to a ``local'' symmetry $\rho^{\rm loc}: \fg[z,z^{-1}] \to {\rm SympVect}(V((z)))$.
Geometrically, a $\fg$-valued function $x(z)$ acts pointwise (with respect to the formal disk) on the formal loop space.
(Although the algebraic loop space is not symplectic in a strict sense,
there is still a local symmetry as $\fg[z,z^{-1}]$ maps to vector fields on the algebraic loop space.)
This action factors through a map $H_\rho^{\rm loc}: \fg[z,z^{-1}] \to \cO(V((z)))$,
and hence is Hamiltonian.
Thus the loop algebra $\fg[z,z^{-1}]$ arises naturally as symmetries of a classical holomorphic field theory.

When we try to quantize, we run into an interesting phenomenon:
the Hamiltonian action $H_\rho^{\rm loc}$ does not extend as a Lie algebra map into the Weyl algebra $W[V]$,
although it makes sense as a linear map.
The failure to respect the brackets determines, however, a cocycle on the loop algebra
by
\[
\alpha_V( x \otimes f(z) , y \otimes g(z)) = \Tr_V(xy) \Res(f \, \partial_z g).
\]
\owen{Check conventions.}
There is thus a central extension of the loop algebra 
\[
\CC c \to \widehat{\fg} \to \fg[z,z^{-1}]
\]
so that 
\[
[x \otimes f, y \otimes g] = [x,y] \otimes fg + \alpha_V( x \otimes f , y \otimes g) c,
\]
and there is a Lie algebra map
\[
\widehat{H}^{\rm loc}_\rho: \widehat{\fg} \to W[V],
\]
sometimes called {\em free field realization} of the affine Lie algebra~$\widehat{\fg}$.
This relationship --- and its extension to a map between the vertex algebras associated to $\widehat{\fg}$ and $V$ --- is a modest \owen{better adjective?} extract of the rich dialogue that has developed between complex analysis, representation theory, and the physics of chiral conformal field theory.
(To get Riemann surfaces into the game, we will need a richer framework than ordinary algebra.)

\owen{Discuss modules?}

\subsection{Algebras in higher dimensions }

The story we have just told admits a natural extension to higher complex dimensions,
but it requires a first step in the direction of derived geometry.

Let $V$ denote a complex vector space (no longer required to be symplectic).
The naive idea is to replace the algebraic loop space $\Map(\CC^\times,  V)$ with a higher-dimensional analog $\Map(\CC^d -\{0\}, V)$.
If one considers ordinary algebraic or holomorphic maps for $d > 1$, 
then every such map extends across the origin, by Hartogs' lemma,
so this seems uninteresting.
One should notice, however, that punctured affine space $\mathring{\AA}^d = \CC^d -\{0\}$ is, as a scheme, not affine, 
and so the {\em derived} global sections of the structure sheaf $\cO_{\rm alg}$ are interesting:
\[
\RR \Gamma(\pAA^d, \cO_{\rm alg}) = H^* (\mathring{\AA}^d, \cO_{\rm alg}) 
= \begin{cases} 
0, & * \neq 0, d-1 \\ 
\CC[z_1,\ldots,z_d], & * = 0 \\ 
\CC[z_1^{-1},\ldots,z_d^{-1}] \frac{1}{z_1 \cdots z_d}, & * = d-1 
\end{cases},
\]
where we are providing a natural basis for the cohomology 
that aims to emphasize analogies with the $d=1$ case.
When $d=1$, note that this computation recovers the Laurent polynomials,
so when $d > 1$, 
we view the cohomology in degree $d-1$ as providing the derived replacement of the polar part of the Laurent polynomials.
Checking this computation is a straightforward exercise in algebraic geometry;
for instance, use the cover by the affine opens of the form $\AA^d \setminus \{z_i =0\}$.
\owen{I want to discuss the module structure on the $d-1$-component.}

A similar result holds in analytic geometry, of course,
so that we have a natural map
\[
\RR \Gamma(\pAA^d, \cO_{\rm alg}) \to \RR \Gamma(\pAA^d, \cO_{\rm an}) \simeq \Omega^{0,*}(\pAA^d)
\]
that replaces our inclusion of Laurent polynomials into the Dolbeault complex on~$\pAA^d$.

There is a natural analog of the residue map as well.
Just as $\CC^\times$ deformation-retracts onto the unit circle,
the punctured affine space $\pAA^d$ retracts onto the unit sphere $S^{2d-1}$.
The Bochner-Martinelli kernel 
\[
\omega_{BM} = \frac{(d-1)!}{(2 \pi i)^d} \frac{1}{(z\zbar)^d} \sum_{i=1}^d (-1)^{i-1} \zbar_i \d \zbar_1 \wedge \cdots \wedge \Hat{\d \zbar_i} \wedge \cdots \wedge \d \zbar_d
\]
determines a cohomology class of degree $d-1$, and we identify it with $1/z_1 \cdots z_d$ in our basis above,
so that $\omega_{BM}$ generates its cohomology group under multiplication by polynomials in the $z_j^{-1}$.
When $d = 1$, note that $\omega_{BM} = (1/2\pi i)\d z/z$, the kernel in the Cauchy residue formula.
It is also quick to check that
\[
\omega_{BM} \wedge \d z_1 \wedge \cdots \wedge \d z_d
\]
pulls back to a volume form on the unit sphere $S^{2d-1}$.
Hence, we define a residue map
\[
\Res: \RR \Gamma(\pAA^d, \cO_{\rm alg})  \to \CC
\]
by
\[
\Res(\alpha) = \oint_{S^{2d-1}} \alpha \wedge \d z_1 \wedge \cdots \wedge \d z_d.
\]
When $d =1$, this formula reduces to the usual residue map.
\owen{That's a bit misleading, maybe, but the integral really does only care about the cohomology class.}

The ingredients are thus in place to mimic the construction of the Weyl algebra for the formal loop space, of the affine Lie algebras, and of free field realization, but now with $\pAA^d$ in place of $\pAA^1 = \CC^\times$.
Reversing our treatment above, we first discuss the higher Kac-Moody algebras, 
following Faonte-Hennion-Kapranov \cite{FHK},
before discussing holomorphic field theory.

It is important to have an explicit dg commutative algebra that models the derived global sections,
and not just the cohomology groups.
There is a beautiful, explicit dg model $A_d$ for the algebraic version derived global sections due to Faonte-Hennion-Kapranov \cite{FHK} and based on the Jouanolou method for resolving singularities. 
In fact, they provide a model $A^{p,*}_d$ for the algebraic $p$-forms as well.
We will gloss their definition, referring the interested reader to their lucid paper for details.

Let $a_d$ denote the algebra  
\[
\CC[z_1,\ldots,z_d, z_1^*,\ldots,z_d^*][(z z^*)^{-1}]
\]
defined by localizing the polynomial algebra with respect to $zz^* = \sum_i z_i z^*_i$.
View this algebra $a_d$ as concentrated in bidegree $(0,0)$.
Consider the bigraded-commutative algebra $R^{*,*}_d$ over $a_d$ that is freely generated in bidegree $(1,0)$ by elements
\[
\d z_1,\ldots , \d z_d,
\] 
and in bidegree $(0,1)$ by
\[
\d z_1^*,\ldots, \d z_d^*.
\]
Note the close resemblance with the Dolbeault complex $\Omega^{*,*}(\CC^d)$ by thinking of $z_j^*$ as $\zbar_j$.
There is a subalgebra $A^{*,*}_d$ inside $\omega \in R^{p,m}_d$ and a natural pair of differentials $\partial$ and $\dbar$ that behave like the corresponding differentials in the Dolbeault complex.
We denote the subcomplex with $p=0$~by 
\[
(A_d, \dbar) = (\bigoplus_{q = 0}^d A_d^{q}[-q], \dbar),
\] 
and it has the structure of a dg commutative algebra.
For $p>0$, the complex $A^{p,*}_d = (\oplus_q A^{p,q}[-q], \dbar)$ is a dg module for $(A_d, \dbar)$.

The following proposition summarizes key properties of the dg algebra $A_d$ and its dg modules $A_{d}^{p,*}$,
by aggregating several results of~\cite{FHK}.

\begin{prop}[\cite{FHK}, Section 1]
\label{prop: Ad} $\;$
\begin{enumerate}
\item
The dg commutative algebra $(A_d,\dbar)$ is a model for $\RR \Gamma(\pAA^d, \sO_{\rm alg})$:
\[
A_d \simeq \RR\Gamma(\pAA^{d}, \sO^{alg}) .
\]
Similarly, $(A_d^{p,*},\dbar) \simeq \RR \Gamma(\pAA^{d}, \Omega^{p}_{\rm alg})$.
\item There is a dense map of commutative bigraded algebras
\[
\jou : A^{*,*}_d \to \Omega^{*,*}(\CC^d \setminus \{0\}),
\]
and the map intertwines with the $\dbar$ and $\partial$ differentials on both sides.
\item There is a unique $\GL_d$-equivariant residue map
\[
{\rm Res}_{z=0} : A_d^{d,d-1} \to \CC
\]
such that for any $\omega \in A^{d,d-1}_d$,
\[
{\rm Res}_{z=0} (\omega) = \oint_{S^{2d-1}} \jou(\omega)
\]
where $S^{2d-1}$ is any sphere centered at the origin in $\CC^d$. 
\item There is a representative $\omega_{BM}^{\rm alg}$ of the Bochner-Martinelli kernel such that
\[
\Res_{z=0} \left(f(z) \omega_{BM}^{alg}(z,z^*) \d z_1 \cdots \d z_d\right) = f(0)
\]
for any $f (z) \in \CC[z_1,\ldots,z_d]$. 
\end{enumerate}
\end{prop}

We can use this algebra and its nice properties to define a kind of ``sphere algebra'' $\Map(\pAA^d,\fg)$ of algebraic maps from the punctured affine space (or the punctured formal disk) into $\fg$,
just as $\fg[z,z^{-1}]$ provides a kind of loop algebra.

\begin{dfn}
For a Lie algebra $\fg$, the {\em sphere algebra} in complex dimension $d$ is the dg Lie algebra~$A_d \otimes \fg$.
Following \cite{FHK} we denote it by~$\fg^\bullet_d$.
\end{dfn}

There are natural central extensions of this sphere algebra as {\em $L_\infty$ algebras},
parallel to the affine Lie algebras appearing as central extensions of the loop algebra.
An $L_\infty$ algebra is a generalization of a Lie algebra;
it is graded vector space $L$ equipped with a countable collection of ``higher brackets'' $[-]_k: L^{\otimes k} \to L$ that satisfy a system of equations generalizing the Jacobi condition for a standard Lie algebra.
For an overview (and an actual definition!) we recommend \owen{add citations}.

In our setting of $\fg^\bullet_d$, Faonte-Hennion-Kapranov introduced central extensions that only affect the $d$-bracket.
For any $\fg$-invariant degree $d+1$ polynomial  $\theta \in \Sym^{d+1}(\fg^*)^\fg$,  the FHK  cocycle is
\[
\label{fhk cocycle}
\begin{array}{cccc}
\theta_{\rm FHK} : & (A_d \tensor \fg)^{\tensor (d+1)} & \to & \CC\\ 
& a_0 \otimes \cdots \otimes a_d & \mapsto & \Res_{z=0} \theta(a_0,\partial a_1,\ldots,\partial a_d)
\end{array}.
\]
This cocycle has cohomological degree $2$ and so determines an unshifted central extension as $L_\infty$ algebras
\[
\CC c \to \widetilde{\fg}^\bullet_{d, \theta} \to \fg^\bullet_d ,
\]
so that the usual bracket (the 2-bracket) of $\fg^\bullet_d$ is unchanged but there is now a nontrivial $d$-bracket defined by~$\theta_{\rm FHK}$.
They dubbed these the {\em higher Kac-Moody algebras}, 
as they specialize to the affine Lie algebras when~$d =1$.

These $L_\infty$ algebras are not just introduced on a whim.
In \cite{FHK} they are shown to appear naturally in various contexts,
notably in studying the derived moduli of $G$-bundles on a complex varieties of dimension~$d$.
\owen{Other applications?}

In addition to the higher Kac-Moody algebras, there are also natural higher analogs of the Weyl algebra of the loop space.
These appear as the observables for a holomorphic field theory known as a $\beta\gamma$ system.

Let $V$ be a symplectic complex vector space and consider the formal sphere algebra $A^{\wedge}_d \otimes V$,
where $A^{\wedge}_d$ denotes the completion of $A_d$ (i.e., where polynomials in $z_j$ and $z_j^*$ are replaced by formal power series).
\owen{Should we introduce a pithy notation?}
Then there is a dg commutative algebra of functions on this space,
which we denote $\cO(A^{\wedge}_d \otimes  V)$,
and it has a kind of Poisson bracket, arising from the residue pairing on $A^{\wedge}_d$ and the symplectic form on $V$,
just as $\cO(V((z)))$ has a Poisson bracket.
It will not surprise the reader to learn that there is a natural deformation quantization $W_d[V]$ into a dg Weyl algebra.
A physicist would interpret this algebra as the canonical and radial quantization of the theory on~$\CC^d - \{0\}$.

If $V$ is a representation of a Lie algebra $\fg$,
there is a local symmetry 
\[
\fg^\bullet_d \to \cO(A^{\wedge}_d \otimes V),
\]
just as $\fg[z,z^{-1}]$ acts on $\cO(V((z)))$. 
One can ask if this action lifts to a quantum symmetry.
The obstruction is, remarkably, an FHK cocycle:
\[
\alpha_{V,d}(x_1 \otimes f_1, \ldots, x_{d+1} \otimes f_{d+1}) = \Tr_V( x_1 \cdots x_{d+1}) \Res(f_1 \partial f_2 \cdots \partial f_{d+1}).
\]
\owen{Fix conventions}
In this sense, the higher Kac-Moody algebras admit a higher free field realization,
but a richer story is possible once we have a higher version of vertex algebras.
We will return to this topic once we have the rich language of factorization algebras.

\end{document}