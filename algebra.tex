\documentclass[11pt]{amsart}

\usepackage{macros, setspace}

\def\U{{\rm U}}
\def\mc{\mathcal}
\def\mcol{\, | \,}
\def\ot{\otimes}
\def\Disj{\operatorname{Disj}}
\def\Open{\operatorname{Open}}
\def\Vect{\operatorname{Vect}}

%trying this out
%\usepackage[upint]{stix}
%\usepackage{cmupint}

\def\brian#1{{\textcolor{blue!65!red}{BRW: {#1}}}}
\def\owen#1{{\textcolor{green!65!black}{OGG: {#1}}}}


\author{Owen Gwilliam and Brian R. Williams}
\date{\today}

%\title{Part III}

\spacing{1.25}

\begin{document}
%\maketitle

\section{Algebras from holomorphic field theories}

In this section we describe  associative algebras and their ``higher'' generalizations, like $A_\infty$ algebras, related to holomorphic field theories,
following the relationship of associative algebras to quantum mechanics and of vertex algebras to chiral conformal field theory.
A particular focus is upon higher Kac-Moody algebras \cite{FHK},
as these offer a tantalizing direction to explore in search of analogs of the rich connections between representation theory, algebraic geometry, and physics familiar to those who have worked with loop groups.
Later --- and it is a central point of this survey --- we will explain how factorization algebras provide a direct conduit from holomorphic field theories to these algebraic constructions.

\subsection{Algebras in mechanics}

A key feature of quantum mechanics is that the observables (or operators) live in an associative algebra.
In many cases this associative algebra is a deformation of a commutative algebra, typically arising as functions on a manifold or variety.
The quintessential example is the Weyl algebra
\[
\CC \langle x, p \rangle /(xp - px = i \hbar),
\] 
which is generated by observables $x$ (``position'') and $p$ (``momentum'') for a quantum particle moving along a line.
There is a parameter $\hbar$, which if sent to zero, recovers a commutative algebra $\CC [ x, p ]$ of complex-valued polynomial functions on the cotangent bundle $T^* \RR$ of the real line~$\RR$.
This example will be a model for much of what we discuss in this paper.
So far we have ignored a lot of features of quantum mechanics (e.g., $\ast$-structures, Hilbert spaces, unitarity) that are important in physics,
and we will continue to do so.

There are some features of this example that we would like to foreground. 
First, the commutative algebra arises as functions on a {\em symplectic} space,
which is the usual mathematical setting for classical mechanics (aside from more subtle situations that require Poisson geometry).
Second, the symplectic form $\omega = \d x \wedge \d p$ equips this algebra 
with a Poisson bracket where $\{ x, p \} = 1$,
which controls the deformation to the Weyl algebra: following Dirac, we promote the Poisson relation to a commutator relation.
These two features motivate the deformation quantization problem: 
given a symplectic (or Poisson) manifold,
describe deformations of its commutative algebra of functions to an associative algebra with the requirement that, to first order,  the commutator recovers the Poisson bracket.
Thanks to Kontsevich \cite{KonDQ}, there is a beautiful answer to this question, which has spawned a mountain of fascinating mathematics (see, e.g., \owen{what?} as a starting place).

Our view on field theory is motivated by this perspective on quantum theory (we discuss it further in Section~\owen{ref the general overview}), 
and it might help the reader to bear in mind a variant of this question:
given a classical holomorphic field theory, what are the natural deformations of its algebra of functions?
Many of the results we discuss later can be seen as generalizations of phenomena associated with deformation quantization.

In particular, consider how symmetries appear in mechanics.
Let $X$ be a symplectic manifold encoding the ``phase space'' of a classical mechanical system,
and let $\{-,-\}$ denote the Poisson bracket on $C^\infty(X)$.
If a Lie group $G$ acts on $X$ by symplectomorphisms (i.e., respects the symplectic structure),
there is a map of Lie algebras $\rho: \fg \to {\rm SympVect}(X)$ so that an ``infinitesimal symmetry'' $x \in \fg$ acts by a symplectic vector field.
In many cases, such infinitesimal symmetries are realized as observables: 
there is a map of Lie algebras $H_\rho: \fg \to C^\infty(X)$ such that $\{ H_\rho(x), - \} = \rho(x)$.
One says that each element $x$ has a Hamiltonian function $H_\rho(x)$ whose associated Hamiltonian vector field is $\rho(X)$.
Classic examples include the momenta for $T^* \RR^n$ arising from translation and rotational symmetry.

How does this set-up fit into the deformation quantization view?
First, note that the free commutative algebra $\Sym(\fg)$ has a canonical Poisson bracket by defining $ \{x, y \}   = [x, y]$ for generators $x, y \in \fg$ and extending by the Leibniz rule.
Hence the Hamiltonian map $H_\rho$ extends to a Poisson map $H_\rho: \Sym(\fg) \to C^\infty(X)$.
We can then ask: does this map quantize? That is, can we compatibly deform the domain and range of the maps to interesting associative algebras as well as deforming to a map of associative algebras?
Here it might be useful to recognize that the enveloping algebra $U\fg$ is a natural deformation quantization of $\Sym(\fg)$, thanks to the Poincar\'e-Birkhoff-Witt theorem.
Thus our question might be reformulated to asking whether there is an associative algebra map $H^q_\rho: U\fg \to C^\infty(X)^q$, where $C^\infty(X)^q$ is a deformation quantization, such that $H^q_\rho$ recovers $H_\rho$ in the classical limit.
Such a map is sometimes called a {\em quantum moment map},
as $H_\rho$ is the pullback of (polynomial) functions along a moment map~$\mu_\rho: X \to \fg^*$.

This class of questions appears throughout mathematics,
and it has played a key role in geometric representation theory and the theory of $D$-modules, where $D$ denotes a ring of differential operators.
After all, for any manifold $X$, the algebra of differential operators $D_X$ is a natural deformation quantization of functions on $T^* X$.
Hence, for any $G$-manifold $X$, it is natural to ask whether there is a representation $\fg \to D_X$ deforming the canonical Poisson representation.
In this spirit, we will search for analogues of the kind of mathematics that has grown out of results by Beilinson-Bernstein \owen{cite}, Kashiwara \owen{cite}, and others.

\subsection{A first step in the holomorphic direction}



\subsection{Algebras in higher dimensions }




\end{document}