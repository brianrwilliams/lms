\documentclass[11pt]{amsart}

\usepackage{macros, setspace}

\def\U{{\rm U}}
\def\mc{\mathcal}
\def\mcol{\, | \,}
\def\ot{\otimes}
\def\Disj{\operatorname{Disj}}
\def\Open{\operatorname{Open}}
\def\Vect{\operatorname{Vect}}

%trying this out
%\usepackage[upint]{stix}
%\usepackage{cmupint}

\def\brian#1{{\textcolor{blue!65!red}{BRW: {#1}}}}
\def\owen#1{{\textcolor{green!65!black}{OGG: {#1}}}}


\author{Owen Gwilliam and Brian R. Williams}
\date{\today}

\title{Part III}

\spacing{1.25}

\begin{document}
\maketitle

\section{Vertex algebras and higher-dimensional analogs \owen{Crappy title}}

So how do factorization algebras appear in holomorphic field theories?

\subsection{Holomorphic factorization algebras}

We restrict ourselves to theories defined on complex affine space.
Let 
\beqn\label{eqn:affine}
\U(n) \ltimes \RR^{2n} \subset {\rm Aff}(\RR^{2n}) .
\eeqn
be the unitary affine group. 
A holomorphic factorization algebra on $\CC^n$ is a factorization algebra with an action by the unitary affine group where the anti-holomorphic translations act trivially (at least up to homotopy).

The (complexified) Lie algebra of \eqref{eqn:affine} is $\mathfrak{gl}(n) \ltimes \CC^{2n}$
with $\CC^{2n}$ spanned by the holomorphic and anti-holomorphic translations $\{\partial_{z_i}, \partial_{\zbar_i}\}$. 
A homotopical way to encode holomorphicity is to introduce the {\em dg} Lie algebra $\CC^{n}_{\rm hol}$ which in degree zero is the $2n$-dimensional space spanned by all translations $\{\partial_{z_i}, \partial_{\zbar_i}\}$, and in degree $(-1)$ is an $n$-dimensional space spanned by elements $\{\Bar{\eta}_i\}$. 
The differential is simply $\d \Bar{\eta}_i = \partial_{\zbar_i}$. 
Its immediate to see that $\CC^n_{\rm hol}$ is quasi-isomorphic to the abelian Lie algebra spanned by the {\em holomorphic} translations $\{\partial_{z_i}\}$. 

\begin{dfn}
An $n$-dimensional {\em holomorphic factorization algebra} is a smoothly~\footnote{\brian{recall smooth}} affine equivariant factorization algebra $\cF$ on $\CC^n$ together with an extension of the complex Lie algebra of infinitesimal symmetries
to the dg Lie algebra $
\mathfrak{gl}(n) \ltimes \CC^{n}_{\rm hol}$. 
\end{dfn}

As an example, consider the `enveloping' type factorization algebra which is built out of the Dolbeault complex on $\CC^n$; it assigns to an open set $U \subset \CC^n$ the cochain complex
\[
{\rm Sym} \left(\Omega^{0,\bu}_c(U)\right) .
\]
This factorization algebra is clearly affine equivariant. 
Trivialization for the anti-holomorphic translation $\frac{\partial}{\partial \zbar_i}$ is given by the contraction of a Dolbeault form by the corresponding vector field:
\beqn\label{eqn:triv}
\Bar{\eta}_i = \frac{\partial}{\partial \zbar_i} \vee (-).
\eeqn

\subsection{Chiral conformal field theories and vertex algebras}

\owen{I want to discuss how VOAs relate to factorization algebras, notably our CDO project, Si's results, and Brueggeman's thesis. We should also gesture at the Beilinson-Drinfeld approach.}

\subsection{Holomorphic $\sigma$-models and higher-dimensional chiral differential operators}

From a geometric perspective, holomorphic $\sigma$-models form an extremely natural class of holomorphic field theories defined in arbitrary complex dimensions. 

We turn our attention to the holomorphic $\sigma$-model of maps $\CC^n$ to the $(n-1)$-shifted cotangent bundle $\T^*[n-1] Y$ of a complex manifold $X$. 
Being a holomorphic AKSZ model, this theory a priori depends on the holomorphic volume form on $\CC^n$; but there is a 
\brian{not sure if this is discussed elsewhere}
\[
{\rm Map}(\CC^n, \T^*[n-1] X) \simeq \T^*[-1] {\rm Map}(\CC^n, X) .
\]


\subsection{Higher Kac-Moody algebras and their free field realizations}

\subsection{Work with Ingmar \owen{not the title}}

\end{document}