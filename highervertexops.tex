\documentclass[11pt]{amsart}

\usepackage{macros, setspace}

\spacing{1.1}

\usepackage{parskip}
\setlength{\parindent}{18pt}
\setlength{\parindent}{0cm}


\author{Owen Gwilliam and Brian R. Williams}
\date{\today}
\title{Holomorphic field theories and higher algebra}

\def\g{{\mathfrak g}}
\def\U{{\rm U}}
\def\mc{\mathcal}
\def\mcol{\, | \,}
\def\ot{\otimes}
\def\Disj{\operatorname{Disj}}
\def\Open{\operatorname{Open}}
\def\Vect{\operatorname{Vect}}
\def\jou{{\mathtt{j}}}
\def\bgamma{{\mathbf{\gamma}}}
\def\bbeta{{\mathbf{\beta}}}
\def\CDO{{\cC\cD}}
\def\SU{{\rm SU}}
\def\sfa{\mathsf{a}}
\def\del{\partial}
\def\til{\Tilde}
\def\Sol{\text{Sol}}
\def\Spec{\text{Spec}}
\def\C{{\mathbb{C}}}
\def\RHom{{\rm RHom}}

\renewcommand{\op}{\operatorname}
\def\lie#1{\ensuremath{\mathfrak{#1}}}

\def\brian#1{{\textcolor{blue!65!red}{BW: {#1}}}}
\def\owen#1{{\textcolor{violet!65!black}{OG: {#1}}}}

%\usepackage[upint]{stix}

\begin{document}

The goal here is to record higher-dimensional analogs of the ingredients for defining a {\em field} in a holomorphic theory (i.e., a higher-dimensional {\em vertex operator}). We do not aspire (yet!) to give the higher-dimensional version of a vertex algebra.

{\it Notations for spaces, functions, and distributions}

Let $\cO_d$ denote a (preferably ``small'' or convenient) model of the derived sections of the structure sheaf (of {\em algebraic} functions) on the punctured affine space $\mathring{\AA}^d$.\\
{\tiny When $d=1$, it is the Laurent polynomials~$\CC[z,z^{-1}]$.}

Let $\widehat{\cO}_d$ denote a (preferably ``small'' or convenient) model of the derived sections of the structure sheaf (of {\em algebraic} functions) on the punctured formal disk $\mathring{\DD}^d$.\\
{\tiny When $d=1$, it is the Laurent series $\CC((z))$.}

Let $\cD_d$ denote a model of the derived ``distributions'' on the punctured affine space $\mathring{\AA}^d$. It is the linear dual to~$\cO_d$.\\
{\tiny When $d=1$, it is $\CC[[z,z^{-1}]]$, which is a notation that indicates infinite formal sums in both directions. The pairing is via the residue pairing. (This notation is poor because it suggests an algebra, but it is not.)}

Let $\widehat{\cD}_d$ denote a model of the derived ``distributions'' on the punctured formal disk $\mathring{\DD}^d$. It is the linear dual to~$\widehat{\cO}_d$.\\
{\tiny When $d=1$, it is again $\CC((z))$, via the residue pairing.}

There are analogs of these constructions for products of affine or formal spaces with the fat diagonal removed.
We will write here the two-variable version.

Consider the space 
\[
\mathring{\AA}^{d}[2]= (\AA^d \times \AA^d) - \Delta_f{[2]}, 
\]
where $\Delta[2]_f$ denotes the fat diagonal
\[
\AA^d \times \{0\} \cup \{0\} \times \AA^d \cup \AA^{2d}_\Delta
\]
with
\[
 \AA^{2d}_\Delta = \{ (x,y) \in \AA^d \times \AA^d \; : \; x \neq y\}.
\]
Let $\mathring{\DD}^d[2]$ denote the corresponding formal space.

Let $\cO_d{[2]}$ denote a model of the derived sections of the structure sheaf (of {\em algebraic} functions) on $\mathring{\AA}^d[2]$.

Let $\widehat{\cO}_d{[2]}$ denote a model of the derived sections of the structure sheaf (of {\em algebraic} functions) on $\mathring{\DD}^d[2]$.

Let $\cD_d[2]$ denote a model of the derived distributions on $\mathring{\AA}^d[2]$.

Let $\widehat{\cD}_d[2]$ denote a model of the derived distributions on $\mathring{\DD}^d[2]$.
\newpage

{\it Operators}

Let $V$ be a cochain complex, which we will treat as a {\em state space}.

A {\em field} is an element of 
\[
\RHom(V \otimes \widehat{\cO}_d, V) \cong \RHom(V, V \widehat{\otimes} \widehat{\cD}_d).
\]
When $d=1$, this definition becomes the usual version: $\Hom(V, V((z)))$.

Note that a field can be seen as a distribution that eats a function on the formal disk and gives a linear operator on $V$ because
\[
\RHom(V \otimes \widehat{\cO}_d, V) \cong \RHom(\widehat{\cO}_d, \RHom(V,V)).
\]
This corresponds to a traditional notion of a ``quantum field'' as an operator-valued distribution.

For notation, if $z$ denotes a coordinate $(z_1,\ldots,z_d)$ for the formal disk $\DD^d$,
we write $a(z)$ for a field using that coordinate.

Given two fields $a(z)$ and $b(w)$, we have a commutator $[a(z),b(w)]$ that eats a function in $\widehat{\cO}_d[2]$ and returns an endomorphism of~$V$.

Two fields are {\em mutually local} if there exists a natural number $N \in \NN$ and an element $f(z,w)$ of the ideal $(z_i-w_i)_{i =1}^d \subset \widehat{\cO}(\DD^d\times \DD^d)$ such that $f(z,w)[a(z),b(w)]$ is cohomologous to zero (i.e., it is exact).

This condition recovers the usual notion of locality after passing to cohomology. 
\owen{I suspect there's something better to say here, informed by your work with Nik. I'm least happy with this definition.}

It is possible to formulate a representative of the ``delta function'' and one can hope that the usual formulas (e.g., describing the structure of the singular parts of commutators or OPEs) have analogs up to exact terms.

{\it Free field realizations}

We want to show that the $L_\infty$-action of a higher Kac-Moody algebra on the state space of a free theory is via mutually local fields.









\end{document}