\documentclass[11pt]{amsart}

\usepackage{macros, setspace}

\spacing{1.1}

\usepackage{parskip}
\setlength{\parindent}{18pt}
\setlength{\parindent}{0cm}


\author{Owen Gwilliam and Brian R. Williams}
\date{\today}
\title{Holomorphic field theories and higher algebra}

\def\g{{\mathfrak g}}
\def\U{{\rm U}}
\def\mc{\mathcal}
\def\mcol{\, | \,}
\def\ot{\otimes}
\def\Disj{\operatorname{Disj}}
\def\Open{\operatorname{Open}}
\def\Vect{\operatorname{Vect}}
\def\jou{{\mathtt{j}}}
\def\bgamma{{\mathbf{\gamma}}}
\def\bbeta{{\mathbf{\beta}}}
\def\CDO{{\cC\cD}}
\def\SU{{\rm SU}}
\def\sfa{\mathsf{a}}
\def\del{\partial}
\def\til{\Tilde}
\def\Sol{\text{Sol}}
\def\Spec{\text{Spec}}
\def\C{{\mathbb{C}}}
\def\RHom{{\rm RHom}}
\newcommand{\sff}{\mathsf{f}}
\newcommand{\sfe}{\mathsf{e}}
%\def\sfR{\mathsf{R}}
%\def\sfA{\mathsf{A}}

\renewcommand{\op}{\operatorname}
\def\lie#1{\ensuremath{\mathfrak{#1}}}

\def\brian#1{{\textcolor{blue!65!red}{BW: {#1}}}}
\def\owen#1{{\textcolor{violet!65!black}{OG: {#1}}}}

%\usepackage[upint]{stix}

\begin{document}

\section{Overview of definitions and goalds}

The goal here is to record higher-dimensional analogs of the ingredients for defining a {\em field} in a holomorphic theory (i.e., a higher-dimensional {\em vertex operator}). We do not aspire (yet!) to give the higher-dimensional version of a vertex algebra.

{\it Notations for spaces, functions, and distributions}

Let $\cO_d$ denote a (preferably ``small'' or convenient) model of the derived sections of the structure sheaf (of {\em algebraic} functions) on the punctured affine space $\mathring{\AA}^d$.\\
{\tiny When $d=1$, it is the Laurent polynomials~$\CC[z,z^{-1}]$.}

Let $\widehat{\cO}_d$ denote a (preferably ``small'' or convenient) model of the derived sections of the structure sheaf (of {\em algebraic} functions) on the punctured formal disk $\mathring{\DD}^d$.\\
{\tiny When $d=1$, it is the Laurent series $\CC((z))$.}

Let $\cD_d$ denote a model of the derived ``distributions'' on the punctured affine space $\mathring{\AA}^d$. It is the linear dual to~$\cO_d$.\\
{\tiny When $d=1$, it is $\CC[[z,z^{-1}]]$, which is a notation that indicates infinite formal sums in both directions. The pairing is via the residue pairing. (This notation is poor because it suggests an algebra, but it is not.)}

Let $\widehat{\cD}_d$ denote a model of the derived ``distributions'' on the punctured formal disk $\mathring{\DD}^d$. It is the linear dual to~$\widehat{\cO}_d$.\\
{\tiny When $d=1$, it is again $\CC((z))$, via the residue pairing.}

There are analogs of these constructions for products of affine or formal spaces with the fat diagonal removed.
We will write here the two-variable version.

Consider the space 
\[
\mathring{\AA}^{d}[2]= (\AA^d \times \AA^d) - \Delta_f{[2]}, 
\]
where $\Delta[2]_f$ denotes the fat diagonal
\[
\AA^d \times \{0\} \cup \{0\} \times \AA^d \cup \AA^{2d}_\Delta
\]
with
\[
 \AA^{2d}_\Delta = \{ (x,y) \in \AA^d \times \AA^d \; : \; x \neq y\}.
\]
Let $\mathring{\DD}^d[2]$ denote the corresponding formal space.

Let $\cO_d{[2]}$ denote a model of the derived sections of the structure sheaf (of {\em algebraic} functions) on $\mathring{\AA}^d[2]$.

Let $\widehat{\cO}_d{[2]}$ denote a model of the derived sections of the structure sheaf (of {\em algebraic} functions) on $\mathring{\DD}^d[2]$.

Let $\cD_d[2]$ denote a model of the derived distributions on $\mathring{\AA}^d[2]$.

Let $\widehat{\cD}_d[2]$ denote a model of the derived distributions on $\mathring{\DD}^d[2]$.
\newpage

{\it Operators}

Let $V$ be a cochain complex, which we will treat as a {\em state space}.

A {\em field} is an element of 
\[
\RHom(V \otimes \widehat{\cO}_d, V) \cong \RHom(V, V \widehat{\otimes} \widehat{\cD}_d).
\]
When $d=1$, this definition becomes the usual version: $\Hom(V, V((z)))$.

Note that a field can be seen as a distribution that eats a function on the formal disk and gives a linear operator on $V$ because
\[
\RHom(V \otimes \widehat{\cO}_d, V) \cong \RHom(\widehat{\cO}_d, \RHom(V,V)).
\]
This corresponds to a traditional notion of a ``quantum field'' as an operator-valued distribution.

For notation, if $z$ denotes a coordinate $(z_1,\ldots,z_d)$ for the formal disk $\DD^d$,
we write $a(z)$ for a field using that coordinate.

Given two fields $a(z)$ and $b(w)$, we have a commutator $[a(z),b(w)]$ that eats a function in $\widehat{\cO}_d[2]$ and returns an endomorphism of~$V$.

Two fields are {\em mutually local} if there exists a natural number $N \in \NN$ and an element $f(z,w)$ of the ideal $(z_i-w_i)_{i =1}^d \subset \widehat{\cO}(\DD^d\times \DD^d)$ such that $f(z,w)[a(z),b(w)]$ is cohomologous to zero (i.e., it is exact).

This condition recovers the usual notion of locality after passing to cohomology. 
\owen{I suspect there's something better to say here, informed by your work with Nik. I'm least happy with this definition.}

It is possible to formulate a representative of the ``delta function'' and one can hope that the usual formulas (e.g., describing the structure of the singular parts of commutators or OPEs) have analogs up to exact terms.

{\it Free field realizations}

We want to show that the $L_\infty$-action of a higher Kac-Moody algebra on the state space of a free theory is via mutually local fields.

\newpage

\section{Models for functions and distributions on punctured affine space}

We want a convenient and explicit model for the commutative algebra-up-to-homotopy $\RR \Gamma(\mathring{\AA}^2)$.
Our approach follows the approach of Keyou Zeng but is written in a more algebraic style.
We quickly recall his results before introducing our version.

\subsection{Zeng's hyperplane approach}

The punctured affine space $\mathring{\AA}^2$ contains a natural real hypersurface given by the unit sphere $S^3 = \{|w_1|^2 + |w_2|^2 = 1\}$, with the $w_i$ being holomorphic coordinates on $\AA^2$.
The theory of CR manifolds provides an analogue of the Dolbeault complex, 
and Zeng works with polynomial functions and polynomial forms within it:\footnote{Zeng claims to compute the CR cohomology but he in fact just works with the direct sum of the $SU(2)$-representations therein, and hence he works with this algebraic subcomplex instead.}
\beqn
\Omega^{0,\bu}_b (S^3) = \big(\C[w_i,\Bar{w}_i, \xi_i] \slash (w_i \Bar{w}_i = 1, w_i \xi_i = 0) \big) .
\eeqn
where the $\xi_i$ have degree 1 (and represent $\d \Bar{w}_i$) and where the differential is $\dbar_b = \xi_i \del_{\Bar{w}_i}$.
\owen{I'm not sure why it must provide a model for $\RR \Gamma(\mathring{\AA}^2)$, but I give some heuristic reasoning. Maybe Zeng gives a reference or proof?}
Note that one can see that polynomials in the coordinates, both holomorphic $w_i$ and antiholomorphic $\Bar{w}_i$, span the complex-valued polynomial functions on $\CC^2 \cong \RR^4$.
Their restriction to the unit-sphere $S^3$ wholly determines their behavior as functions on affine space, and hence also on punctured affine space. 
The elements $\xi_i$ correspond to $d\Bar{w}_i$ in the Dolbeault complex of punctured affine space.
The differential $\dbar_b$ then imposes the constraint, in degree zero, that these functions be holomorphic,
and its image is the image of $\dbar$ intersected with polynomial  forms in $d \Bar{w}_i$.
Thus the cohomology of this complex $\Omega^{0,\bu}_b (S^3)$ is a model for $\RR \Gamma(\mathring{\AA}^2)$.

Zeng analyzes this complex using the natural $SU(2)$-action on the 3-sphere, which is the restriction of the fundamental representation to this hypersurface.
More accurately, $SU(2)$ acts on the linear span of $w_1$ and $w_2$ by the fundamental representation and on the linear span of $\Bar{w}_1$ and $\Bar{w}_2$ by the conjugate (or anti-fundamental) representation, 
and it respects the relation $\sum_i w_i \Bar{w}_i = 1$ and the differential $\dbar_b$, 
so the action descends to the whole complex.
One can decompose the complex into irreducible representations, 
allowing one to identify the cohomology as a subcomplex.
This construction yields an explicit deformation retraction of the $\dbar_b$-complex onto its cohomology.

\subsection{An algebraic version}

We mimic the construction but replace the role of $SU(2)$ by $SL(2)$.
\owen{I mostly left things alone below, except for removing typos or modestly rewording things I found confusing.}


Let $\sfR$ be the commutative algebra generated by variables $z_i,\lambda_i,i=1,2$ subject to the relation $\sum_i z_i \lambda_i = 1$.
Let $\sfA$ be the graded commutative algebra freely generated over $\sfR$ by a degree $+1$ element $\omega$.
Thus $\sfA = \sfR \oplus \sfR \omega [-1]$.
Equip it with the differential
\[
\dbar(z_i) = 0 \quad\text{and}\quad \dbar(\lambda_i) = \epsilon_{ij} z_j \omega
\]
extended as a derivation.
That is, $\dbar(\lambda_1) = z_2 \omega$ and $\dbar(\lambda_1) = -z_1 \omega$.

We view the $z_i$ as the $w_i$ in Zeng's complex, and the $\lambda_i$ as the $\Bar{w}_i$.
The element $\omega$ corresponds to $\Bar{w}_1\xi_2 - \Bar{w}_2\xi_1$.
In fact, the key result of this section is the following.

\begin{prop}
There is a $SU(2)$-equivariant isomorphism of dg commutative algebras
\beqn
\Phi \colon \sfA \to \Omega^{0,\bu}_b (S^3) 
\eeqn
defined by $\Phi(z_i) = w_i$, $\Phi(\lambda_i) = \Bar{w}_i$, and $\Phi (\omega) = \ep_{ij} \Bar{w}_i \xi_j$.
\end{prop}

This proposition guarantees that we have a well-behaved model for~$\RR \Gamma(\mathring{\AA}^2)$.

The algebra $\sfR$ is naturally an $SL(2)$ representation where the linear span $V_z$ of $z_1$, $z_2$ transforms in the fundamental representation  and the linear span $V_\lambda$ of  $\lambda_1$, $\lambda_2$ transforms in the dual (or contragredient) representation.
In particular, with respect to the standard Cartan subgroup, $z_1$ and $\lambda_2$ have weight 1 and $z_2$ and $\lambda_1$ have weight~$-1$.

Consider now the space of all polynomials
\[
\C[z_i,\lambda_i] = \Sym(V_z \oplus V_\lambda) \cong \Sym(V_z) \otimes \Sym(V_\lambda).
\]
The Lie algebra $\sl(2)$ acts by the lowering and raising operators
\beqn
\sff = z_2 \del_{z_1} - \lambda_1 \del_{\lambda_2} \quad\text{and}\quad \sfe = z_1 \del_{z_2} - \lambda_2 \del_{\lambda_1}.
\eeqn
The space of polynomials breaks up into a direct sum over the subrepresentations 
\[
V_{a,b} = \Sym^a(V_z) \otimes \Sym^b(V_\lambda)
\]
of polynomials that are homogenous of order $a$ in the variables $z_i$ and homogenous of order $b$ in the variables~$\lambda_i$.

Recall that $\Sym^a(V_z)$ is an irreducible representation of dimension $a+1$ (and hence highest weight $a$) and $\Sym^b(V_\lambda)$ is an irreducible representation of dimension $b+1$ (and hence highest weight $a$).
Let $I_m$ denote the irreducible representation of highest weight $m$ (and hence of dimension $m+1$).
By standard facts about $SL(2)$, we see that $V_{a,b}$ decomposes as a direct sum of irreducible representations
\[
V_{a,b} \cong I_{a+b} \oplus I_{a+b-2} \oplus \cdots \oplus I_{a-b}
\]
for $a \geq b$. (The other situation can be seen by symmetry.)

There is a more concrete way to describe this decomposition.
Consider the differential operator
\beqn
\triangle = \del_{z_1} \del_{\lambda_1} + \del_{z_2} \del_{\lambda_2}
\eeqn
acting on $\C[z_i,\lambda_i]$.
\owen{This isn't the image of the quadratic Casimir. In what sense is it the relevant Laplacian? On the other hand, it is a weight zero operator, even if it lowers the bigrading by (1,1).}
Let $H_{a,b} \subset V_{a,b}$ denote the harmonic polynomials of bidegree $(a,b)$.
Examples of harmonic polynomials include $z_1^a \lambda_2^b$ for arbitrary $a,b$ and
$z_1 \lambda_1 - z_2 \lambda_2$ for $a = 1 = b$.
A direct computation shows that there is a direct sum decomposition \owen{of representations?}
\begin{align*}
V_{a,b} & = H_{a,b} \oplus |(z,\lambda)|^2 V_{a-1,b-1} \\
& = H_{a,b} \oplus |(z,\lambda)|^2 H_{a-1,b-1} \oplus |(z,\lambda)|^4 H_{a-2,b-2} \oplus \cdots 
\end{align*}
(Compare with the well-known construction of spherical harmonics.)

Consider now  
\[
|(z,\lambda)|^2 \define z_1 \lambda_1 + z_2 \lambda_2, 
\]
which describes the evaluation pairing. 
It is preserved by the $SL(2)$-action, and hence we see that the canonical quotient map 
\[
q \colon \C[z_i,\lambda_i] \to R = \C[z_i,\lambda_i]/(|(z,\lambda)|^2 = 1)
\]
is $SL(2)$-equivariant.
Moreover, inside $V_{a,b}$, any multiple of $|(z,\lambda)|^2$ is annihilated by $q$,
so $R$ is spanned by the images of $H_{a,b}$.
Thus, there is a direct sum decomposition $\sfR = \bigoplus_{a,b} \sfR_{a,b}$ where $\sfR_{a,b} = q(H_{a,b})$ is the image of the subspace of harmonic polynomials in $\C[z_i,\lambda_i]$ under the quotient map.

\begin{eg}
The space $\sfR_{a,0}$ consists of polynomials in $z_1,z_2$ of homogenous degree~$a$.
The space $\sfR_{0,b}$ consists of polynomials in $\lambda_1,\lambda_2$ of homogenous degree $b$.
The space $\sfR_{1,1}$ is three-dimensional with a basis given by
\beqn
z_1 \lambda_2, z_1 \lambda_1 - z_2 \lambda_2, z_2 \lambda_1 .
\eeqn
\end{eg}

The following fact is quite useful for us.

\begin{lem}
Each $\sfR_{a,b}$ is isomorphic to the irreducible $SL(2)$-representation of dimension $(a+b+1)$.
Its highest weight vector is $z_1^a \lambda_2^b \in \sfR_{a,b}$, and a basis for $\sfR_{a,b}$ is obtained as
\beqn
\{z_1^a \lambda_2^b, \sff (z_1^a \lambda_2^b), \sff^2 (z_1^a \lambda_2^b), \ldots, \sff^{a+b} (z_1^a \lambda_2^b) \} .
\eeqn
\end{lem}

This lemma is proved by applying the description of the decomposition into irreducibles.

By direct inspection, one finds that the $\dbar$-operator is $SL(2)$-equivariant.

\begin{cor}
The differential $\dbar$ is an isomorphism from $\sfR_{a,b}$ to $\sfR_{a+1,b-1} \omega$ for $b > 0$.
Thus, the cohomology is 
\[
H^0(A) = \bigoplus_{a \in \NN} \sfR_{a,0}
\]
and
\[
H^1(A) = \bigoplus_{b \in \NN} \sfR_{0,b}.
\]
\end{cor}

\begin{proof}
For the first claim, note that the map is equivariant and nontrivial when $b > 0$. Thus by Schur's lemma it is an isomorphism.
The description of cohomology follows by keeping the summands $\sfR_{a,b}$ that appear in the kernel (for $H^0$) and that do not appear in the image (for $H^1$).
\end{proof}

%Define the involution $\Bar{(-)}$ on $\sfR$ by $\zbar_i = \lambda_i$ and consider the inner product on $\sfR$ given by the formula
%\beqn
%\<a,b\> = \op{Res} \left(\Bar{a} b \omega \right).
%\eeqn
%This inner product is $SU(2)$ invariant.
%
%\begin{eg}
%An orthonormal basis for $\sfR_{a,0}$ is
%\beqn
%e_m^{(a,0)} = \sqrt{\frac{(a+1)!}{m! (a-m)!}} z_1^m z_2^{a-m} , \quad m = 0,1,\ldots,a .
%\eeqn
%Likewise, an orthonormal basis for $\sfR_{0,b}$ is
%\beqn
%e_m^{(0,b)} = \sqrt{\frac{(a+1)!}{m! (a-m)!}} (-\lambda_1)^m \lambda_2^{a-m} , \quad m = 0,1,\ldots,b .
%\eeqn
%An orthonormal basis for $\sfR_{1,1}$ is
%\beqn
%e_2^{(1,1)} = \sqrt{6} z_1 \lambda_2, \quad e_1^{(1,1)} = \sqrt{3}(z_1 \lambda_1 - z_2 \lambda_2), \quad e_0^{(1,1)} = \sqrt{6} z_2 \lambda_1 .
%\eeqn
%\end{eg}

\begin{proof}[Proof of proposition]
In fact $\Phi$ maps the subspace $\sfR_{a,b}$ isomorphically to $\cH_{a,b}$ where $\cH_{a,b} \subset \Omega^{0,\bu}_b (S^3)$ is the space of harmonic polynomials that are of bidegree $(a,b)$ in the variables $(w_i,\Bar{w}_j)$.
Now observe that
\beqn
\Phi(\dbar(\lambda_i)) = - \Phi(\ep_{ij} z_j \omega) = - \ep_{ij} w_j \Phi (\omega) = \xi_i = \dbar_b \Phi (\lambda_i) 
\eeqn
to see that the differentials are compatible.
\end{proof}

\subsection{Homotopy data}

Let $\sfH \define H^\bu(\sfA)$ and let $i,p$ be the $SU(2)$-equivariant inclusions and projections $i \colon \sfH_2 \hookrightarrow \sfA$, $p \colon \sfA \to \sfH$, given by the direct sum description of the corollary above.

There is a canonical contracting homotopy $H$ on $\sfA$ 
because for each $b > 0$, there is a canonical equivariant inverse $\sfR_{a,b} \to \sfR_{a+1,b-1} \omega$.
Explicitly, send the highest weight vector $z_1^{a+1} \lambda_2^{b-1} \omega$ to the highest weight vector $z_1^{a} \lambda_2^{b}$ and then determine where other vector go via the lowering operator.
For $b = 0$, $H|_{\sfR_{0,b}\omega} = 0$.

\owen{Check the side conditions!}

\begin{lem}
The data $(i,p,H)$ determines a special deformation retraction where $i$ is a map of dg commutative algebras.
\end{lem}

Note that $p$ is not a map of algebras. 
For example,
\beqn
p(z_1 \lambda_1) = \frac12
\eeqn
as $z_1 \lambda_1$ admits a decomposition
\beqn
z_1 \lambda_1 = \frac12 (z_1 \lambda_1 - z_2 \lambda_2) + \frac12 .
\eeqn

\subsection{Residues}

Define
\beqn
\op{Res} \colon \sfA \to \C[1]
\eeqn
by the rules that $\op{Res}|_{\sfA^0} = 0$ and
\beqn
\op{Res} \left(f(z,\lambda) \omega \right) = p (f(z,\lambda)) (z=0) .
\eeqn

\begin{lem} The following are true regarding the higher algebraic residue:
\begin{enumerate}
\item The residue is $\lie{sl}(2)$ invariant.
\item For $f \in \C[z_1,z_2]$ one has $\op{Res}(f \omega) = f(0)$.
\item For any $f \in \sfA$ one has $\op{Res}(\dbar f) = 0$.
\end{enumerate}
\end{lem}
\begin{proof}
Item (1) holds since $p$ is the $\lie{sl}(2)$-equivariant projection. 
For (2), note that for $f \in \C[z_1,z_2]$ that $p(f) = f$.

Finally, observe that for $f = z_1^a \lambda_2^b$ we have
\beqn
\op{Res}(\dbar f) = \op{Res}(b z_1^{a+1} \lambda_2^{b-1}) .
\eeqn
For $b =0$ this expression clearly vanishes. 
For $b = 1$ the right hand side is $p( z_1^{a+1})(0) = 0$ for all $a \geq 0$.
Finally, for $b > 1$ one has $p(z_1^{a+1} \lambda_2^{b-1}) = 0$ for all $a \geq 0$.
Using $\lie{sl}(2)$ invariance this proves (3).
\end{proof}

\subsection{Distributions on punctured affine space}

The model $\sfA$ for algebraic functions on punctured affine space was constructed using the ring
\beqn
\sfR = \C[z_i,\lambda_i] \slash (z_i \lambda_i = 1) .
\eeqn
Let $\Hat{\sfR}$ denote the completion of this ring with respect to the ideal generated by $z_1,z_2$.
So I think
\beqn
\Hat{\sfR} = \C[[z_i]][\lambda_i] \slash (z_i \lambda_i=1) .
\eeqn

Let $\Hat{\sfA}$ be the graded algebra free generated over $\Hat{\sfR}$ and the degree one variable $\omega$.
The cohomology of $(\Hat\sfA, \dbar)$ is $\C[[z_i]]$ in degree zero and $\C[\lambda_i]$ in degree one.

\section{Recollection on anomalies for quantum moment maps}

Let $V$ be a dg vector space and let $V^* = \Hom(V, \CC)$ denote its dual complex. 
Consider the dg Poisson algebra $\cP = \Sym(V \oplus V^*)$ whose Poisson bracket extends the skew-symmetrized evaluation pairing on $V \oplus V^*$.
We view $\cP$ as functions on the cotangent bundle~$T^* V$.
Observe that the quadratic functions form a dg Lie subalgebra
\[
\cP_2 = (\Sym^2(V \oplus V^*), \{-,-\})
\]
of $\cP$.

There is a natural deformation quantization of $\cP$ as a Weyl algebra $\cW$.
Observe that there is a dg Lie subalgebra
\[
\cW_2 = (\Sym^2(V \oplus V^*) \oplus \Sym^0(V \oplus V^*), [-,-])
\]
of $\cW$.
It is a central extension of $\cP_2$, and let $\omega \in H^2(\cP_2)$ denote the associated 2-cocycle.

Suppose that there is a Lie algebra map $J: \fg \to \cP_2$.
We view as a special class of symmetries for the free theory encoded by $\cP$:
these are the symmetries where $\fg$ deforms the system to another {\em free} system.
This situation appears in the free field realization we are studying.

\begin{lem}
Given a Lie algebra map $J: \fg \to \cP_2$, the obstruction to lifting it to a map of Lie algebras $\fg \to \cW_2$ is the 2-cocycle $J^* \omega \in H^2(\fg)$.
\end{lem}

We call $J^*\omega$ the {\em anomaly} for quantizing this symmetry.

\begin{cor}
If the anomaly $J^*\omega$ does not vanish, then there is an $L_\infty$ algebra map $J^q: \widehat{\fg} \to \cW_2$ out of the $L_\infty$ extension
\[
\CC \to \widehat{\fg} \to \fg
\]
determined by $J^* \omega$.
\end{cor}



\end{document}