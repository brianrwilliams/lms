\documentclass[11pt]{amsart}

\usepackage{macros, setspace}

\spacing{1.1}

\usepackage{parskip}
\setlength{\parindent}{18pt}
\setlength{\parindent}{0cm}


\author{Owen Gwilliam and Brian R. Williams}
\date{\today}
\title{Holomorphic field theories and higher algebra}

\def\g{{\mathfrak g}}
\def\U{{\rm U}}
\def\mc{\mathcal}
\def\mcol{\, | \,}
\def\ot{\otimes}
\def\Disj{\operatorname{Disj}}
\def\Open{\operatorname{Open}}
\def\Vect{\operatorname{Vect}}
\def\jou{{\mathtt{j}}}
\def\bgamma{{\mathbf{\gamma}}}
\def\bbeta{{\mathbf{\beta}}}
\def\CDO{{\cC\cD}}
\def\SU{{\rm SU}}
\def\sfa{\mathsf{a}}
\def\del{\partial}
\def\til{\Tilde}
\def\Sol{\text{Sol}}
\def\Spec{\text{Spec}}
\def\C{{\mathbb{C}}}
\def\RHom{{\rm RHom}}
%\def\sfR{\mathsf{R}}
%\def\sfA{\mathsf{A}}

\renewcommand{\op}{\operatorname}
\def\lie#1{\ensuremath{\mathfrak{#1}}}

\def\brian#1{{\textcolor{blue!65!red}{BW: {#1}}}}
\def\owen#1{{\textcolor{violet!65!black}{OG: {#1}}}}

%\usepackage[upint]{stix}

\begin{document}

The goal here is to record higher-dimensional analogs of the ingredients for defining a {\em field} in a holomorphic theory (i.e., a higher-dimensional {\em vertex operator}). We do not aspire (yet!) to give the higher-dimensional version of a vertex algebra.

{\it Notations for spaces, functions, and distributions}

Let $\cO_d$ denote a (preferably ``small'' or convenient) model of the derived sections of the structure sheaf (of {\em algebraic} functions) on the punctured affine space $\mathring{\AA}^d$.\\
{\tiny When $d=1$, it is the Laurent polynomials~$\CC[z,z^{-1}]$.}

Let $\widehat{\cO}_d$ denote a (preferably ``small'' or convenient) model of the derived sections of the structure sheaf (of {\em algebraic} functions) on the punctured formal disk $\mathring{\DD}^d$.\\
{\tiny When $d=1$, it is the Laurent series $\CC((z))$.}

Let $\cD_d$ denote a model of the derived ``distributions'' on the punctured affine space $\mathring{\AA}^d$. It is the linear dual to~$\cO_d$.\\
{\tiny When $d=1$, it is $\CC[[z,z^{-1}]]$, which is a notation that indicates infinite formal sums in both directions. The pairing is via the residue pairing. (This notation is poor because it suggests an algebra, but it is not.)}

Let $\widehat{\cD}_d$ denote a model of the derived ``distributions'' on the punctured formal disk $\mathring{\DD}^d$. It is the linear dual to~$\widehat{\cO}_d$.\\
{\tiny When $d=1$, it is again $\CC((z))$, via the residue pairing.}

There are analogs of these constructions for products of affine or formal spaces with the fat diagonal removed.
We will write here the two-variable version.

Consider the space 
\[
\mathring{\AA}^{d}[2]= (\AA^d \times \AA^d) - \Delta_f{[2]}, 
\]
where $\Delta[2]_f$ denotes the fat diagonal
\[
\AA^d \times \{0\} \cup \{0\} \times \AA^d \cup \AA^{2d}_\Delta
\]
with
\[
 \AA^{2d}_\Delta = \{ (x,y) \in \AA^d \times \AA^d \; : \; x \neq y\}.
\]
Let $\mathring{\DD}^d[2]$ denote the corresponding formal space.

Let $\cO_d{[2]}$ denote a model of the derived sections of the structure sheaf (of {\em algebraic} functions) on $\mathring{\AA}^d[2]$.

Let $\widehat{\cO}_d{[2]}$ denote a model of the derived sections of the structure sheaf (of {\em algebraic} functions) on $\mathring{\DD}^d[2]$.

Let $\cD_d[2]$ denote a model of the derived distributions on $\mathring{\AA}^d[2]$.

Let $\widehat{\cD}_d[2]$ denote a model of the derived distributions on $\mathring{\DD}^d[2]$.
\newpage

{\it Operators}

Let $V$ be a cochain complex, which we will treat as a {\em state space}.

A {\em field} is an element of 
\[
\RHom(V \otimes \widehat{\cO}_d, V) \cong \RHom(V, V \widehat{\otimes} \widehat{\cD}_d).
\]
When $d=1$, this definition becomes the usual version: $\Hom(V, V((z)))$.

Note that a field can be seen as a distribution that eats a function on the formal disk and gives a linear operator on $V$ because
\[
\RHom(V \otimes \widehat{\cO}_d, V) \cong \RHom(\widehat{\cO}_d, \RHom(V,V)).
\]
This corresponds to a traditional notion of a ``quantum field'' as an operator-valued distribution.

For notation, if $z$ denotes a coordinate $(z_1,\ldots,z_d)$ for the formal disk $\DD^d$,
we write $a(z)$ for a field using that coordinate.

Given two fields $a(z)$ and $b(w)$, we have a commutator $[a(z),b(w)]$ that eats a function in $\widehat{\cO}_d[2]$ and returns an endomorphism of~$V$.

Two fields are {\em mutually local} if there exists a natural number $N \in \NN$ and an element $f(z,w)$ of the ideal $(z_i-w_i)_{i =1}^d \subset \widehat{\cO}(\DD^d\times \DD^d)$ such that $f(z,w)[a(z),b(w)]$ is cohomologous to zero (i.e., it is exact).

This condition recovers the usual notion of locality after passing to cohomology. 
\owen{I suspect there's something better to say here, informed by your work with Nik. I'm least happy with this definition.}

It is possible to formulate a representative of the ``delta function'' and one can hope that the usual formulas (e.g., describing the structure of the singular parts of commutators or OPEs) have analogs up to exact terms.

{\it Free field realizations}

We want to show that the $L_\infty$-action of a higher Kac-Moody algebra on the state space of a free theory is via mutually local fields.

\newpage

\section{A model for punctured affine space}

Let $\sfR$ be the commutative algebra generated by variables $z_i,\lambda_i,i=1,2$ subject to the relation $\sum_i z_i \lambda_i = 1$.
Recall that $\sfA$ is the graded commutative algebra freely generated over $\sfR$ by a degree $+1$ element $\omega$.
Thus $\sfA = \sfR \oplus \sfR \omega [-1]$.

The algebra $\sfR$ is naturally an $SU(2)$ representation where the $(z_1,z_2)$ transforms in the fundamental representation and $(\lambda_1,\lambda_2)$ transforms in the anti-fundamental representation.
Explicitly, the raising and lowering operators are
\beqn
f = z_2 \del_{z_1} - \lambda_1 \del_{\lambda_2}, \quad e = z_1 \del_{z_2} - \lambda_2 \del_{\lambda_1} .
\eeqn

Let $V_{a,b} \subset \C[z_i,\lambda_i]$ be the space of all polynomials which are homogenous of order $a$ in the variables $z_i$ and homogenous of order $b$ in the variables $\lambda_i$.
Consider the Laplacian operator
\beqn
\triangle = \del_{z_1} \del_{\lambda_1} + \del_{z_2} \del_{\lambda_2}
\eeqn
and denote $H_{a,b} \subset V_{a,b}$ the harmonic polynomials of bidegree $(a,b)$.
There is an orthogonal decomposition 
\begin{align*}
V_{a,b} & = H_{a,b} \oplus |(z,\lambda)|^2 V_{a-1,b-1} \\
& = H_{a,b} \oplus |(z,\lambda)|^2 H_{a-1,b-1} \oplus |(z,\lambda)|^4 H_{a-2,b-2} \oplus \cdots 
\end{align*}
where $|(z,\lambda)|^2 = z_1 \lambda_1 + z_2 \lambda_2$.
In particular, after imposing the relation $|(z,\lambda)|^2 = 1$ we see that every element in $\sfR$ can be written as a sum of elements in the images of $H_{a,b}$ under the natural projection $p \colon \C[z_i,\lambda_i] \to \sfR$.
We therefore have a decomposition $\sfR = \oplus \sfR_{a,b}$ where $\sfR_{a,b} = p(H_{a,b})$ is the image of the subspace of harmonic polynomials in $\C[z_i,\lambda_i]$ under the quotient map.

Each $\sfR_{a,b}$ is isomorphic to the irreducible $(a+b+1)$-dimensional $SU(2)$ representation where the highest weight vector is
\beqn
z_1^a \lambda_2^b \in \sfR_{a,b} .
\eeqn
A basis for $\sfR_{a,b}$ is obtained as
\beqn
\{z_1^a \lambda_2^b, f (z_1^a \lambda_2^b), f^2 (z_1^a \lambda_2^b), \ldots, f^{a+b} (z_1^a \lambda_2^b) \} .
\eeqn
Note that the $\dbar$-operator acts on the subspaces $\sfR_{a,b}$ in the following way
\beqn
\dbar \colon \sfR_{a,b} \to \sfR_{a+1,b-1} \omega .
\eeqn

\begin{eg}
The space $\sfR_{a,0}$ consists of polynomials in $z_1,z_2$ of homogenous degree~$a$.
The space $\sfR_{0,b}$ consists of polynomials in $\lambda_1,\lambda_2$ of homogenous degree $b$.
The space $\sfR_{1,1}$ is three-dimensional with a basis given by
\beqn
z_1 \lambda_2, z_1 \lambda_1 - z_2 \lambda_2, z_2 \lambda_1 .
\eeqn
\end{eg}

Define the involution $\Bar{(-)}$ on $\sfR$ by $\zbar_i = \lambda_i$ and consider the inner product on $\sfR$ given by the formula
\beqn
\<a,b\> = \op{Res} \left(\Bar{a} b \omega \right).
\eeqn
This inner product is $SU(2)$ invariant.

\begin{eg}
An orthonormal basis for $\sfR_{a,0}$ is
\beqn
e_m^{(a,0)} = \sqrt{\frac{(a+1)!}{m! (a-m)!}} z_1^m z_2^{a-m} , \quad m = 0,1,\ldots,a .
\eeqn
Likewise, an orthonormal basis for $\sfR_{0,b}$ is
\beqn
e_m^{(0,b)} = \sqrt{\frac{(a+1)!}{m! (a-m)!}} (-\lambda_1)^m \lambda_2^{a-m} , \quad m = 0,1,\ldots,b .
\eeqn
An orthonormal basis for $\sfR_{1,1}$ is
\beqn
e_2^{(1,1)} = \sqrt{6} z_1 \lambda_2, \quad e_1^{(1,1)} = \sqrt{3}(z_1 \lambda_1 - z_2 \lambda_2), \quad e_0^{(1,1)} = \sqrt{6} z_2 \lambda_1 .
\eeqn
\end{eg}

Recall that 
\beqn
\Omega^{0,\bu}_b (S^3) = \big(\C[w_i,\Bar{w}_i, \xi_i] \slash (w_i \Bar{w}_i = 1, w_i \xi_i = 0), \quad \dbar_b = \xi_i \del_{\Bar{w}_i} \big) .
\eeqn

\begin{prop}
There is a (dense) $SU(2)$-equivariant isomorphism of commutative dg algebras
\beqn
\Phi \colon \sfA_2 \to \Omega^{0,\bu}_b (S^3) 
\eeqn
defined by $\Phi(z_i) = w_i$, $\Phi(\lambda_i) = \Bar{w}_i$, and $\Phi (\omega) = \ep_{ij} \Bar{w}_i \xi_j \define \epsilon$.
\end{prop}
\begin{proof}
In fact $\Phi$ maps the subspace $\sfR_{a,b}$ isomorphically to $\cH_{a,b}$ where $\cH_{a,b} \subset \Omega^{0,\bu}_b (S^3)$ is the space of harmonic polynomials which are of bidegree $(a,b)$ in the variables $(w_i,\Bar{w}_j)$.
To see that the differentials are compatible we observe that
\beqn
\Phi(\dbar(\lambda_i)) = - \Phi(\ep_{ij} z_j \omega) = - \ep_{ij} w_j \epsilon = \xi_i = \dbar_b \Phi (\lambda_i) .
\eeqn
\end{proof}

\subsection{Homotopy data}

Let $\sfH \define H^\bu(\sfA)$ and let $i,p$ be the obvious $SU(2)$ equivariant inclusions and projections $i \colon \sfH_2 \hookrightarrow \sfA_2$, $p \colon \sfA_2 \to \sfH_2$.

\begin{rmk}
The inclusion $i$ is a map of commutative dg algebras.
Note that $p$ is not a map of algebras; for example
\beqn
p(z_1 \lambda_1) = \frac12
\eeqn
as $z_1 \lambda_1$ admits an orthogonal decomposition
\beqn
z_1 \lambda_1 = \frac12 (z_1 \lambda_1 - z_2 \lambda_2) + \frac12 .
\eeqn
\end{rmk}

Define the degree $-1$ operator 
\beqn
H \colon \sfA_2 \to \sfA_2 [-1]
\eeqn
by the condition that $H|_{\sfR_{0,b}\omega} = 0$ and 
\beqn
H(z_1^a \lambda_2^b) = \frac{1}{(b+1)} z_1^{a-1} \lambda_2^{b+1} .
\eeqn
Extend this to all of $\sfA_2$ using $SU(2)$ equivariance.
Note that for $a > 0$ this is an operator 
\beqn
H \colon \sfR_{a,b} \omega \to \sfR_{a-1,b+1} .
\eeqn

\begin{prop}
The triple $(p,i,H)$ defines a special deformation retract between $\sfA$ and its cohomology.
\end{prop}


\section{Distributions on punctured affine space}

The model $\sfA$ for algebraic functions on punctured affine space was constructed using the ring
\beqn
\sfR = \C[z_i,\lambda_i] \slash (z_i \lambda_i = 1) .
\eeqn
Let $\Hat{\sfR}$ denote the completion of this ring with respect to the ideal generated by $z_1,z_2$.
So I think
\beqn
\Hat{\sfR} = \C[[z_i]][\lambda_i] \slash (z_i \lambda_i) .
\eeqn

Let $\Hat{A}$ be the graded algebra free generated over $\Hat{\sfR}$ and the degree one variable $\omega$.
The cohomology of $(\sfA, \dbar)$ is $\C[[z_i]]$ in degree zero and $\C[\lambda_i]$ in degree one.







\end{document}